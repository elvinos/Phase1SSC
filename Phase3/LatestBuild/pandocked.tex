\section{Introduction}\label{introduction}

This report analyses the effectiveness of a using a PID controller
derived in Phase 2. Before implementing the controller on the Quanser
rig, the controller was first tested by placing nonlinearities into the
simulation in the form of a rate limiter and saturation. Through
applying the design process the new control objectives were met which
could them be implemented on the Quanser rig and further refined to
create a controller which met requirements. The following report
discusses both the design process, results and any observations made
through creating an effective controller for the Quanser rig.

The following transfer function was used taken from phase 1:

\begin{align*}
&\text{$2^{nd}$ Order: }k \cdot \frac { 1.109\cdot \frac{180} {\pi} }{ s^2 + 0.1313s +1.109 }
&&\text{$1^{st}$ Order: }k \cdot \frac { 1 }{ 15.24s +1 }
\end{align*}

\section{Theory Transfer Function Control
Refinement}\label{theory-transfer-function-control-refinement}

Clearly give the transfer function, control architecture and steps of
the design process. Provide a table with the controller gains obtained
from the theoretical part, from the initial tuning in Simulink, and
those finally used in the Quanser. Show the closed-loop output responses
for the plant and controller. Overlay in a single plot the responses
from {[}2{]} and {[}3{]} (using the same step and the same initial
condition).

\section{Nonlinearity Simulation Control
Refinement}\label{nonlinearity-simulation-control-refinement}

\section{Quanser Controller}\label{quanser-controller}
